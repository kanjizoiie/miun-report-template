\chapter*{Abstract}
    \addcontentsline{toc}{chapter}{Abstract}
    The abstract acts as a description of the reports contents. This allows for the possibility to have a quick review of the report and provides an overview of the whole report, i.e. contains everything from the objectives and methods to the results and conclusions. Examples: “The objective of this study has been to answer the question…. The study has been conducted with the aid of…. The study has shown that…” Do not mention anything that is not covered in the report. An abstract is written as one piece and the recommended length is 200-250 words. References to the report's text, sources or appendices are not allowed; the abstract should “stand on its own”. Only use plain text, with no characters in italic or boldface, and no mathematical formulas. The abstract can be completed by the inclusion of keywords; this can ease the search for the report in the library databases.

    \textbf{Keywords:} Reinforcement learning, Autonomous vehicles,
\chapter*{Acknowledgments}
    \addcontentsline{toc}{chapter}{Acknowledgments}
    Acknowledgments or Foreword (choose one of the heading alternatives) are not mandatory but can be applied if you as the writer wish to provide general information about your exam work or project work, educational program, institution, business, tutors and personal comments, i.e. thanks to any persons that may have helped you. Acknowledgments are to be placed on a separate page.

{\normalfont\sffamily
    \tableofcontents
    \addcontentsline{toc}{chapter}{Contents}
}

\chapter*{Terminology / Notation}
    \addcontentsline{toc}{chapter}{Terminology / Notation}
    (Choose one of the headline alternatives.) A possible list of terms, abbreviations and variable names with brief explanations may be placed after the table of contents, but is not required. Note that although a term is explained in the list of terms, it should also be explained in the chapter text where it is used the first time.
    \begin{center}
        \begin{tabularx}{\textwidth}{ X  X }
            Example 1 & Example 1 Description \\ 
            Example 2 & Example 2 Description \\ 
            Example 3 & Example 3 Description \\ 
        \end{tabularx}
    \end{center}